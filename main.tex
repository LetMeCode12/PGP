%%%%%%%%%%%%%%%%%%%%%%%%%%%%%%%%%%%%%%%%%%%%%%%
%%% Template for lab reports used at STIMA
%%%%%%%%%%%%%%%%%%%%%%%%%%%%%%%%%%%%%%%%%%%%%%%

%%%%%%%%%%%%%%%%%%%%%%%%%%%%%% Sets the document class for the document
% Openany is added to remove the book style of starting every new chapter on an odd page (not needed for reports)
\documentclass[11pt,polish, openany]{book}

%%%%%%%%%%%%%%%%%%%%%%%%%%%%%% Loading packages that alter the style
\usepackage[]{graphicx}
\usepackage[]{color}
\usepackage{alltt}
\usepackage[T1]{fontenc}
\usepackage[utf8]{inputenc}
\setcounter{secnumdepth}{3}
\setcounter{tocdepth}{3}
\setlength{\parskip}{\smallskipamount}
\setlength{\parindent}{0pt}

% Set page margins
\usepackage[top=100pt,bottom=100pt,left=68pt,right=66pt]{geometry}

% Package used for placeholder text
\usepackage{lipsum}

% Prevents LaTeX from filling out a page to the bottom
\raggedbottom

% Adding both languages
\usepackage[english, polish]{babel}

% All page numbers positioned at the bottom of the page
\usepackage{fancyhdr}
\fancyhf{} % clear all header and footers
\fancyfoot[C]{\thepage}
\renewcommand{\headrulewidth}{0pt} % remove the header rule
\pagestyle{fancy}

% Changes the style of chapter headings
\usepackage{titlesec}
\titleformat{\chapter}
   {\normalfont\LARGE\bfseries}{\thechapter.}{1em}{}
% Change distance between chapter header and text
\titlespacing{\chapter}{0pt}{50pt}{2\baselineskip}

% Adds table captions above the table per default
\usepackage{float}
\floatstyle{plaintop}
\restylefloat{table}

% Adds space between caption and table
\usepackage[tableposition=top]{caption}

% Adds hyperlinks to references and ToC
\usepackage{hyperref}
\hypersetup{hidelinks,linkcolor = black} % Changes the link color to black and hides the hideous red border that usually is created

% If multiple images are to be added, a folder (path) with all the images can be added here 
\graphicspath{ {Figures/} }

% Separates the first part of the report/thesis in Roman numerals
\frontmatter

%%%%%%%%%%%%%%%%%%%%%%%%%%%%%% Starts the document
\begin{document}

%%% Selects the language to be used for the first couple of pages
\selectlanguage{polish}

%%%%% Adds the title page
\begin{titlepage}
	\clearpage\thispagestyle{empty}
	\centering
	\vspace{1cm}

	% Titles
	% Information about the University
	{\normalsize Uniwersytet Technologiczno-Przyrodniczy \\ 
		im. Jana i Jędrzeja Śniadeckich \\
		w Bydgoszczy \par}
		\vspace{3cm}
	{\Huge \textbf{Projektowanie i programowanie gier}} \\
	%\vspace{1cm}
	%{\large \textbf{xxxxx} \par}
	\vspace{4cm}
	{\normalsize DOKUMENTACJA PROJEKTU GRY GOTHIC \\ % \\ specifies a new line
	             Paweł Rudnicki 110974 \\
	             Norbert Skulski 110797\par}
	\vspace{5cm}
    
    \centering \includegraphics[scale=0.5]{utp.png}
    
    \vspace{0.5cm}
		
	% Set the date
	{\normalsize 12-03-2020 \par}
	
	\pagebreak

\end{titlepage}

% Adds a table of contents
\tableofcontents{}

%%%%%%%%%%%%%%%%%%%%%%%%%%%%%%%%%%%%%%%%%%%%%%%%%%%%%%%%%%%%%%%%%%%%%%%%%%%%%%%%%%%%%%%%%%%%
%%%%%%%%%%%%%%%%%%%%%%%%%%%%%%%%%%%%%%%%%%%%%%%%%%%%%%%%%%%%%%%%%%%%%%%%%%%%%%%%%%%%%%%%%%%%
%%%%% Text body starts here!
\mainmatter

\chapter{Wstępny abstrakt}\label{chapt:sum}
Fabuła gry opowiada historię o bezimiennym bohaterze w kolonii karnej znanej również jako Górnicza Dolina.
Jest ona usytuwana na wyspie Khorinis, która została otoczona magiczną barierą, którą można pokonać
tylko ze strony świata zewnętrznego ponieważ wszelkie próby wydostania się z jej wnętrza kończą się tragicznie.
Magiczna bariera została stworzona przez magów na rozkaz króla Rhobara II w celu zapobiegnięcia kradzieży
magicznej rudy wydobywanej w owej dolinie, którą potrzebował do produkcji broni.
Król Rhobar II pokonał niemal wszystkich wrogów swojego królestwa. Poza orkami. Trzeba rzec, że nigdy nie grzeszyli
inteligencją, lecz ich główną umiejętnością jest formowanie szyków przejawiających się przeogromną siłą, i 
determinacją, które są nieodzowną zaletą w celu osiągnięcia końca gatunku ludzkiego. Historia gry rozpoczyna się, kiedy
główny bohater trafia do Górniczej Doliny jako kolejny, nic nie znaczący jeniec, którym został poprzez narażenie się na drobne
kradzieże wbrew interesu królestwa. Zanim zostanie on zesłany, i przejdzie przez magiczną barierę otrzymuje list od Maga Ognia
zaadresowany do Najwyższego Maga Ognia stacjonującego w tak zwanym Starym Obozie, który jest jednym z trzech głównych obozów rozśanych po
Górniczej Dolinie. Defakto, nie jest to jego jedyny cel jaki otrzymał. Bezimienny bohater chce również uciec z Górniczej Doliny, jak tylko będzie
potrafił.

\includegraphics[scale=0.9]{mapa.jpg}

Świat jaki gracz ma do dyspozycji to już wspomniana wcześniej Górnicza Dolina, w której znajdują się trzy obozy konkurujące ze sobą o wszelakie zasoby pozwalające przetrwać w tym brutalnym świecie oraz wieża tajemniczego maga Xardasa.
\begin{itemize}
    \item [1] \textbf{Stary Obóz} - Najstarszy ze wszystkich obozów, mieszczący się na terenie zamku, w którym niegdyś mieszkała straż królewska, magnaci, magowie ognia oraz kopacze, a który po stworzeniu bariery został przejęty przez więźniów. W odróżnieniu od reszty ten obóz rządzi się surowymi prawami. Posiada kopalnię magicznej rudy. Tylko ten obóz prowadzi wymianę z królem, chociaż Nowy Obóz także posiada swoją kopalnię. Z trzech obozów tylko ten nie planuje zniszczyć bariery, jego celem jest wyłącznie handel z królem i Obozem Bractwa. Dzięki temu jest najbogatszy ze wszystkich. Rhobar II, w zamian za magiczną rudę, oddaje wszystko, czego Stary Obóz zażąda. Stary Obóz otoczony jest drewnianą palisadą i dzieli się na zewnętrzny pierścień, w którym żyją kopacze i cienie, oraz wewnętrzny (zamek), gdzie przebywają strażnicy, magowie ognia i magnaci. Przywódcą Starego Obozu jest magnat o imieniu Gomez. W zewnętrznym pierścieniu znajduje się również arena, gdzie walczą przedstawiciele trzech obozów, a także plac targowy. Inne grupy więźniów również posiadają swoich przywódców. Przywódcą strażników jest Thorus, cieni – Diego, a magów ognia – Corristo.
    \item [2] \textbf{Nowy Obóz} - Obóz powstały jako efekt rozłamu w Starym Obozie. Dąży do zniszczenia magicznej bariery i uwolnienia więźniów poprzez detonację wielkiego kopca rudy. Potrzebny w tym celu surowiec wydobywają w Wolnej Kopalni krety. Większość domostw, siedziba magów wody oraz góra rudy znajduje się w wielkiej jaskini. Członkowie Nowego Obozu to głównie dezerterzy i uciekinierzy ze Starego Obozu. W tym obozie nie ma zasad, a prawo to tylko abstrakcyjne pojęcie. By jednak chronić wielki kopiec magicznej rudy, magowie wody zatrudnili najemników, którzy chronią ich i kopiec przed szkodnikami. Jest to, mimo ogromnej ilości posiadanej rudy, najbiedniejszy obóz, utrzymujący się głównie z napadów na konwoje Starego Obozu, przez co znajdują się o krok od wojny domowej. Dzięki tamie wodnej i odpowiednim warunkom możliwa jest uprawa ryżu, który służy m.in. do produkcji ryżówki – alkoholu sprzedawanego w karczmie. Przywódcą najemników w Nowym Obozie jest Generał Lee, wtrącony do kolonii za morderstwo, którego nie popełnił. Przywódcą szkodników jest Lares, a arcymistrzem magów wody – Saturas.
    \item [3] \textbf{Obóz Bractwa} - Jest to obóz (nazywany również Obozem na Bagnie lub Obozem Sekty), którego członkowie wyrzekli się starych bogów, czcząc Śniącego, który ma im pomóc w odzyskaniu wolności. Dzięki wizji Y'Beriona, duchowego przywódcy bractwa, uwierzyli, że Śniący ma za zadanie pomóc wiernym, niszcząc barierę, a niewiernych ukarać. Obóz Sekty propaguje palenie bagiennego ziela, rosnącego tylko tutaj. Każdy nowicjusz dostaje dzienną porcję, by móc pozostać w duchowym kontakcie ze Śniącym. Niektórzy guru po większej ilości bagiennego ziela mają wizje dotyczące Śniącego. Przez pozostałych więźniów członkowie bractwa są uważani za świrów, czy też wariatów, być może z powodu tatuaży na ciele, dziwnych strojów, całkowicie ogolonych głów lub po prostu odmiennych wierzeń. Wielu więźniów dołącza do tego obozu głównie z powodu bagiennego ziela i jego specjalnych właściwości. Obóz składa się z drewnianych domów stojących na ziemi i zawieszonych na drzewach oraz świątyni, w której mieszka Y'Berion wraz z Natalią i Chani, które najprawdopodobniej są jego kurtyzanami. Mieszkańcy obozu dzielą się na nowicjuszy, których głównym zadaniem jest zbieranie i obróbka bagiennego ziela, guru, którzy otrzymują wizje od Śniącego oraz strażników świątynnych, którzy pilnują bezpieczeństwa, a także zbierają wydzielinę pełzaczy – potworów podobnych do wielkich pająków nękających kopaczy w Starej Kopalni. Ośmiu guru przewodzi obozowi. Najwyższy z nich to Y'Berion, którego zastępcą jest Cor Kalom, a trzecią najważniejszą osobą w całym bractwie jest Cor Angar. Pozostałych pięciu guru to Baal Namib, Baal Cadar, Baal Tyon, Baal Orun i Baal Tondral.
    \item [4] \textbf{Wieża Xardasa} - druga wieża nekromanty Xardasa. Znajduje się w samym środku terytoriów należących do orków, w małej kotlince niedaleko miasta orków. Jest także najbardziej wysuniętym na południe punktem Górniczej Doliny.Xardas wzniósł ją po zniszczeniu poprzedniej wieży z pomocą służących mu ożywieńców. Droga do wieży jest bardzo ciężka i pełna niebezpieczeństw. Dostać się do niej można idąc przez wąski wąwóz, który broniony jest przez kamiennego golema, lodowego golema oraz ognistego golema
\end{itemize}
\chapter{Rozszerzone streszczenie gry}
Akcja gry rozgrywa się na terenie kopalni Korinis osadzonej w Górniczej dolinie.
Ówczesny król Myrtany Robar II toczący wojnę z orkami potrzebował surowców aby dostarczyć swoim wojską wyposażenie. Nakazał więc wysłanie wszystkich skazańców z królestwa do Korinis aby przeprowadzali wydobycie potrzebnych zasobów. By uniemożliwić im ucieczkę nakazał stworzenie magicznej bariery.
Podczas tworzenia magicznej bariery nie wszystko poszło zgodnie z planem i uwięziła ona również magów ja tworzących.
\newline
\includegraphics[scale=0.37]{bariera}
\newline
Bohater gry to więzień o nieznanym imieniu, Zostaje wtrącony do kolonii karnej za nieznane graczowi przestępstwo.  Przed wtrąceniem dostaje od maga zadanie dostarczenia listu do jego pobratymców znajdujących się pod drugiej stronie bariery. Po wtrąceniu na teren bariery przechodzi „Chrzest”. Zostaje pobity przez znajdujących się tam więźniów. Przed śmiercią ratuje go wysoko postawiony łucznik ze starego obozu o imieniu Diego. Od tamtej pory jednym z jego początkowych zadań jest staranie się o przynależność do jednego z obozów. W tym celu Bezimienny musi wykonywać zadania zlecane mu przez ważne osobistości w wybranym obozie.
Po przystąpieniu do którejkolwiek frakcji bohater otrzymuje od jej władz polecenie pomocy Bractwu Śniącego w przygotowaniach rytuału nawiązania kontaktu z Śniącym. Wyznawcy Śniącego wierzą, że gdy dojdzie do przebudzenia ich śpiącego boga, pomoże im on wydostać się z kolonii karnej. Skazaniec otrzymuje prawo do audiencji u przywódcy Bractwa – Y'Beriona. Każe on odnaleźć bohaterowi kamień ogniskujący – artefakt, który był wykorzystywany w procesie tworzenia magicznej bariery.
\newline
\begin{center}
	\includegraphics[scale=0.2]{kognisk}
\end{center}

Po odnalezieniu kamienia alchemik Bractwa imieniem Cor Kalom informuje Bezimiennego, że do przygotowania rytuału potrzebna jest mu wydzielina żyjących w Starej Kopalni groźnych stworzeń zwanych pełzaczami. W tym celu Bractwo regularnie wysyła do kopalni strażników świątynnch, którzy przynoszą Cor Kalomowi wnętrzności zabitych stworzeń. Cor Kalom uważa jednak, że musi istnieć efektywniejsze źródło wydzieliny niż wnętrzności pełzaczy i wysyła bohatera na wyprawę celem znalezienia go. Bohater udaje się do kopalni, gdzie odkrywa lokalizację gniazda stworzeń. Dostaje się do niego, pokonuje ich królową i zabiera jaja pełzaczy do Cor Kaloma. Ostatnią rzeczą potrzebną do przeprowadzenia rytuału jest księga zwana Almanachem – zawiera ona instrukcje wykorzystywania kamienia ogniskującego. Bezimienny znajduje ją w jaskini będącą kryjówką wrogo nastawionych czarnych goblinów.
Uczestnikom rytuału objawia się wizja ruin, w których rezydują orkowie. Tuż po zakończeniu wizji przewodzący ceremonii Y'Berion traci przytomność. Bezimienny zostaje wysłany na pomoc ekspedycji wysłanej w celu zbadania leżącego w pobliżu obozu Bractwa cmentarzyska orków. Na miejscu okazuje się, że ekspedycja zawiodła, a cmentarzysko nie zawiera żadnej wskazówki co do dalszych działań. Bohater wraca do obozu. Wkrótce potem Y'Berion umiera. Tuż przed śmiercią ostrzega przed dalszymi próbami budzenia Śniącego i upatruje nadziei na wolność w planie magów wody z Nowego Obozu. Plan ów polega na detonacji wielkiego kopca magicznej rudy. Na polecenie arcymaga wody imieniem Saturas Bezimienny odnajduje pozostałe kamienie ogniskujące i zostaje wysłany z misją przekonania magów ognia ze Starego Obozu do podjęcia próby zniszczenia bariery.
W międzyczasie dochodzi do podmycia Starej Kopalni przez podziemny zbiornik wodny. Kopalnia ta była źródłem przychodów dla Starego Obozu. Przywódca rządzących obozem magnatów imieniem Gomez rozkazuje przejąć zbrojnie należącą do Nowego Obozu Wolną Kopalnię. Mieszkający w Starym Obozie magowie ognia sprzeciwili się planom władz obozu i zostali wymordowani z rozkazu magnatów. Aby uniknąć odwetu ze strony Nowego Obozu, Gomez nakazuje strażnikom zamknięcie bram oraz atakowanie każdego, kto się do nich zbliża. Jeśli Bezimienny podjął decyzję o zostaniu członkiem Starego Obozu to zostanie z niego usunięty. W wyniku braku możliwości skorzystania z pomocy magów ognia Saturas informuje bohatera o trzynastym magu uczestniczącym w procesie tworzenia bariery o imieniu Xardas i każe go odnaleźć.
\newline
\begin{center}
\includegraphics[scale=0.5]{xardas}
\end{center}

Xardas to nekromanta, który mieszka samotnie w wieży na środku terytorium zajętego przez orków. Po odnalezieniu go bohater dowiaduje się, że ma on własny pomysł na zniszczenie bariery i że plan magów wody jest nieskuteczny. Zleca on Bezimiennemu odnalezienie zbuntowanego orkowego szamana imieniem Ur-Shak, ochronienie go przed tropiącymi go pobratymcami oraz przesłuchanie go. Ur-Shak informuje nas, że Śniący jest demonem przywołanym przed wiekami przez orkowych szamanów z innego wymiaru. Sam zaś Śniący przebywa na dnie skomplikowanego podziemnego kompleksu jaskiń i korytarzy zwanym Świątynią Śniącego.
\newpage Wejście do świątyni znajduje się w Mieście Orków znajdującym się na terytorium tej rasy. Aby bezpiecznie poruszać się po terytorium orków i dotrzeć do ich miasta bohater potrzebuje amuletu o nazwie Ulu-Mulu. 
\newline
\begin{center}
	\includegraphics[scale=0.2]{umumulu}
\end{center}
Pomóc w jego wykonaniu ma Bezimiennemu przyjaciel Ur-Shaka pochwycony przez ludzi jako niewolnik i zmuszony do pracy w Wolnej Kopalni. Zdaniem Xardasa magia Śniącego wpłynęła na proces powstawania bariery i jedynym sposobem na zniszczenie jej jest wygnanie demona. Następnie, bohater wraz z najemnikiem Gornem z Nowego Obozu ruszają na odsiecz Wolnej Kopalni i odzyskują ją z rąk strażników Gomeza. Orkowy niewolnik pomaga Bezimiennemu wykonać amulet.
Po dotarciu do Miasta Orków bohater odnajduje wejście do Świątyni Śniącego i dostaje się do środka. Po drodze pokonuje orkowych szamanów-ożywieńców i zbiera ich oręż, który służy jako klucze do bram broniących dostępu do Śniącego. Ponadto Bezimienny odkrywa dodatkowy starożytny miecz o nazwie Uriziel pilnowany przez jednego z szamanów. Ten jednak jest bezużyteczny. Ostatni napotkany szaman okazuje się niewrażliwy na jakąkolwiek broń lub zaklęcie. Bohater wydostaje się ze świątyni i wraca do wieży Xardasa. Nekromanta opowiada mu historię znalezionego miecza. Zgodnie z sugestią nekromanty bohater wraz z jedynym ocalałym z rzezi magiem ognia wykorzystują kopiec magicznej rudy zgromadzony w Nowym Obozie do naprawy oręża. Bezimienny ściąga na siebie w ten sposób gniew magów wody.
Po ucieczce z Nowego Obozu bohater wraca do Świątyni Śniącego, pokonuje ostatniego szamana i dociera do leża demona. Na miejscu staje naprzeciw obudzonego Śniącego. Używając mieczy orkowych szamanów Bezimienny otwiera portal prowadzący do wymiaru, z którego przybył Śniący. Gra kończy się po tym, jak za pomocą Uriziela Bezimienny wtrąca przeciwnika do innego świata.

\chapter{Projektowanie świata gry}\label{chapt:doe}
\chapter{Projektowanie postaci}\label{chapt:model}
\section{Problem geometry and setup}
\section{Mesh generation and description}
\section{Numerical schemes}

\chapter{Flowboard}\label{chapt:results}
\section{Test 1}
\subsection{Grid convergence}

\chapter{Rozwój fabuły i zmiany poziomów}\label{chapt:results}
\section{Test 1}
\subsection{Grid convergence}

\chapter{Scenariusz}\label{chapt:results}
\section{Test 1}
\subsection{Grid convergence}
\chapter{Uwagi}

\pagebreak


% Adding a bibliography if citations are used in the report
\bibliographystyle{plain}
\bibliography{bibliography.bib}
% Adds reference to the Bibliography in the ToC
\addcontentsline{toc}{chapter}{\bibname}

\pagebreak

\chapter*{Appendix A: Resources}
[\textit{Report the config files of the software used (i.e. SU2 \cite{economon2015su2} and the mesher). Also attach to this report an archive with the mesh files, solutions and the reference solution data (e.g. data points of a Cp plot ...)}]
\section*{Mesh configuration files}
\section*{SU2 configuration files}
% \section{Reference solution data}


\end{document}
